\documentclass[12pt]{book}
%\usepackage[width=4.375in, height=7.0in, top=1.0in, papersize={5.5in,8.5in}]{geometry}
\usepackage[a4paper, top=1.0in, left=1.5in, right=1.25in, bottom=1.25in]{geometry}
\usepackage[pdftex]{graphicx}
\usepackage{amsmath}
\usepackage{amssymb}
\usepackage{tipa}
%\usepackage{txfonts}
\usepackage{textcomp}
%\usepackage{amsthm}
%\usepackage{array}
%\usepackage{xy}
\usepackage{fancyhdr}

\pagestyle{fancy}
\renewcommand{\chaptermark}[1]{\markboth{#1}{}}
\renewcommand{\sectionmark}[1]{\markright{\thesection\ #1}}
\fancyhf{}
\fancyhead[LE,RO]{\bfseries\thepage}
\fancyhead[LO]{\bfseries\rightmark}
\fancyhead[RE]{\bfseries\leftmark}
\renewcommand{\headrulewidth}{0.5pt}
\renewcommand{\footrulewidth}{0pt}
\addtolength{\headheight}{0.5pt}
\setlength{\footskip}{0in}
\renewcommand{\footruleskip}{0pt}
\fancypagestyle{plain}{%
\fancyhead{}
\renewcommand{\headrulewidth}{0pt}
}
%
%\parindent 0in
\parskip 0.05in
%
\begin{document}
\frontmatter
%
\chapter*{\Huge \center Math Notes}
\thispagestyle{empty}
%{\hspace{0.25in} \includegraphics{./ru_sun.jpg} }
\section*{\huge \center Xiong Ding}
\newpage
\subsection*{\center \normalsize Copyright \copyright 2015 by Xiong Ding}
\subsection*{\center \normalsize All rights reserved.}
\subsection*{\center \normalsize ISBN \dots}
\subsection*{\center \normalsize \dots Publications}
%
\chapter*{\center \normalsize To my PhD}
%
\tableofcontents
%
\mainmatter

%%%%%%%%%%%%%%%%%%%%%%%%%%%%%%%%%%%%%%%%%%%%%%%%%%%%%%%%%%%%%%%%%%%%%%
% main content starts
\chapter{Geometry, Topology and Physics -- Nakahara}
\begin{itemize}
\item inclusion map
\item 
  If $f:X\to Y$ preserves these algebraic structures, then $f$ is called a \textbf{homomorphism}.
  If a homomorphism is bijective, then it is called an \textbf{isomorphism}.
\item
  equivalence relation, equivalence classes, quotient space ($X/\sim$), M\"{o}bius strip,
  $T^2$ (2-d torus), $\Sigma_g$ (torus with $g$ handles)
\item
  linear map ( an homomorphism preserving vector addition and scalar multiplication), \\
  linear function ( $V \to K$), \\
  $dim V = dim(ker f) + dim(im f)$.
\item
  dual space ($V^*$) is a functional space whose axes are given by base functions 
  $e^{i}(e_j)=\delta_{ij}$. This functional space is also a vector space, and if $V$ 
  is finite dimensional, we have $dim V = dim V^*$. Basically, $f \in V^*$ means 
  $f$ is linear functional defined in $V$, namely, $f:V\mapsto K$. Given an isomorphism 
  $g : V\mapsto V^*$, define the bilinear form:
  \[
    g(\mathbf{v}, \mathbf{w}) \doteq \langle \mathbf{v}, \mathbf{w} \rangle_{g} 
    = \langle g\mathbf{v}, \mathbf{w} \rangle
    = (g(\mathbf{v}))(\mathbf{w})
  \]
  Note, $g(\mathbf{v})$ returns a linear functional.
\item
  discrete/trivial/usual topology,  Hausdorff spaces (every metric space is Hausdorff space), \\
  closure $\bar{A}$, interior $A^\circ$, boundary $b(A)$, \\
  connected, arcwise connected, simply connected.
\item
  homeomorphism is different from homomorphism. \\
  why homeomorphism ? Because it defines a equivalence relation which enables us to deform one 
  topological space continuously to another topological space in the same class. 
\end{itemize}

\chapter{Grassmann Algebra -- John Browne}

Exterior product
\begin{itemize}
\item 
  cobasis $\underline{e_i}$ defined as $e_i \wedge \underline{e_j} = \delta_{ij}  e_1 \wedge e_2 \cdots \wedge e_n$ \\
  determinant, 
  \[
    \alpha_1 \wedge \alpha_2 \cdots \wedge \alpha_n = Det \cdot 
    e_1 \wedge e_2 \cdots \wedge e_n
  \]
  Cofactors
\item
  simplicity (0, 1, $n-1$ and $n$ are simple)
\item
  exterior division: 
  \[
    \frac{ \alpha_1 \wedge \alpha_2 \cdots \wedge \alpha_n \wedge \beta} {\beta} =
    (\alpha_1 + t_1 \beta) \wedge (\alpha_2 + t_2 \beta)\cdots \wedge (\alpha_n + t_n \beta)
  \]
\end{itemize}

Regressive Product
\begin{itemize}
\item 
  
\end{itemize}

\chapter{Pattern formation outside of equilibrium -- Cross and Hohenberg}
\begin{itemize}
\item 
\end{itemize}

\backmatter
%
\begin{thebibliography}{99}
\bibitem{baba_books}
Books of Shrii Shrii Anandamurti (Prabhat Ranjan Sarkar): \\
http://shop.anandamarga.org/
\bibitem{anandamitra}
Avtk. Ananda Mitra Ac., \emph{The Spiritual Philosophy of Shrii Shrii Anandamurti: A Commentary on Ananda Sutram}, Ananda Marga Publications (1991) \\
ISBN: 81-7252-119-7
\end{thebibliography}
\end{document}
